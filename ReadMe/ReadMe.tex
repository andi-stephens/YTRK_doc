\documentclass[12pt,]{article}
\usepackage{lmodern}
\usepackage{amssymb,amsmath}
\usepackage{ifxetex,ifluatex}
\usepackage{fixltx2e} % provides \textsubscript
\ifnum 0\ifxetex 1\fi\ifluatex 1\fi=0 % if pdftex
  \usepackage[T1]{fontenc}
  \usepackage[utf8]{inputenc}
\else % if luatex or xelatex
  \ifxetex
    \usepackage{mathspec}
  \else
    \usepackage{fontspec}
  \fi
  \defaultfontfeatures{Ligatures=TeX,Scale=MatchLowercase}
\fi
% use upquote if available, for straight quotes in verbatim environments
\IfFileExists{upquote.sty}{\usepackage{upquote}}{}
% use microtype if available
\IfFileExists{microtype.sty}{%
\usepackage{microtype}
\UseMicrotypeSet[protrusion]{basicmath} % disable protrusion for tt fonts
}{}
\usepackage[margin=1in]{geometry}
\usepackage{hyperref}
\hypersetup{unicode=true,
            pdftitle={Stock Assessment Template in R Markdown and Git},
            pdfauthor={Melissa Monk, NMFS SWFSC},
            pdfborder={0 0 0},
            breaklinks=true}
\urlstyle{same}  % don't use monospace font for urls
\usepackage{graphicx,grffile}
\makeatletter
\def\maxwidth{\ifdim\Gin@nat@width>\linewidth\linewidth\else\Gin@nat@width\fi}
\def\maxheight{\ifdim\Gin@nat@height>\textheight\textheight\else\Gin@nat@height\fi}
\makeatother
% Scale images if necessary, so that they will not overflow the page
% margins by default, and it is still possible to overwrite the defaults
% using explicit options in \includegraphics[width, height, ...]{}
\setkeys{Gin}{width=\maxwidth,height=\maxheight,keepaspectratio}
\IfFileExists{parskip.sty}{%
\usepackage{parskip}
}{% else
\setlength{\parindent}{0pt}
\setlength{\parskip}{6pt plus 2pt minus 1pt}
}
\setlength{\emergencystretch}{3em}  % prevent overfull lines
\providecommand{\tightlist}{%
  \setlength{\itemsep}{0pt}\setlength{\parskip}{0pt}}
\setcounter{secnumdepth}{5}
% Redefines (sub)paragraphs to behave more like sections
\ifx\paragraph\undefined\else
\let\oldparagraph\paragraph
\renewcommand{\paragraph}[1]{\oldparagraph{#1}\mbox{}}
\fi
\ifx\subparagraph\undefined\else
\let\oldsubparagraph\subparagraph
\renewcommand{\subparagraph}[1]{\oldsubparagraph{#1}\mbox{}}
\fi

%%% Use protect on footnotes to avoid problems with footnotes in titles
\let\rmarkdownfootnote\footnote%
\def\footnote{\protect\rmarkdownfootnote}

%%% Change title format to be more compact
\usepackage{titling}

% Create subtitle command for use in maketitle
\newcommand{\subtitle}[1]{
  \posttitle{
    \begin{center}\large#1\end{center}
    }
}

\setlength{\droptitle}{-2em}
  \title{Stock Assessment Template in R Markdown and Git}
  \pretitle{\vspace{\droptitle}\centering\huge}
  \posttitle{\par}
  \author{Melissa Monk, NMFS SWFSC}
  \preauthor{\centering\large\emph}
  \postauthor{\par}
  \date{}
  \predate{}\postdate{}

% This file contains all of the LaTeX packages you may need to compile the document
% Documentation for each package can be found onlines
\usepackage{tabularx}                                             % table environment providing flexibility
\usepackage{caption}                                              % for creating captions  
\usepackage{longtable}                                            % allows tables to span multiple pages
\usepackage{rotating}                                             % allows for sideways tables
\usepackage{float}                                                % floating environments; may not need in rmarkdown
\usepackage{placeins}                                             % keeps floats from moving
\usepackage{indentfirst}                                          % indents first paragraph of a section
\usepackage{mdwtab}                                               % continued float multi-page figure
\usepackage{enumerate}                                            % create lists
\usepackage{hyperref}                                             % highlight cross references
\hypersetup{colorlinks=true, urlcolor=blue, linktoc=page, linkcolor=blue, citecolor=blue} %define referencing colors
%\usepackage{makebox}                                             % make boxes around text
\usepackage[usenames,dvipsnames]{xcolor}                          % color name options
%\usepackage[space]{grffile}                                      % spaces in file name path
\usepackage{soul}                                                 % highlight text
\usepackage{enumitem}                                             % numbered lists
\usepackage{lineno}                                               % Line numbers; comment out for final
\usepackage{upquote}                                              % produce grave accent in latex
\usepackage{verbatim}                                             % produces verbatim results
\usepackage{fancyvrb}                                             % verbatim in a box
%\usepackage[inline]{showlabels}                                   % show table and figure labels; comment out for final
%\usepackage{draftwatermark}                                      % places Draft watermark in background; comment out for final
\usepackage{textcomp}                                             % fixes error with packages interfering
\usepackage{lscape}                                               % rotate pages - to allow for landscape longtables




\linenumbers                                                      % specify use of line numbers

\definecolor{light-gray}{gray}{.85}

\begin{document}
\maketitle

{
\setcounter{tocdepth}{3}
\tableofcontents
}
\newpage

\section{Introduction}\label{introduction}

The stock assessment template described is based on the
\href{http://www.pcouncil.org/wp-content/uploads/2015/08/H3_Att3_China_FULLAssmt_E-Only_SEPT2015BB.pdf}{2015
China rockfish assessment document}. It is designed to aid in writing a
stock assessment document, for models using Stock Synthesis and the R
package \href{https://github.com/r4ss/r4ss}{r4ss}. The template can
currently handle up to three independent assessment models within the
same assessment document, e.g., China rockfish was split into three
assessments, 1) south of \(40^\circ 10^\prime\) N. latitude, 2)
\(40^\circ 10^\prime\) N. latitude to the OR/WA border, and 3)
Washington state.

The current repository contains .csv, SS, and r4ss files for an
arbitrary stock assessment. Removing these files before you replace them
with your own will result in runtime errors. It is recommended that you
replace the files as you prepare them, knitting and committing the
document each time you make a significant change.

Portions of the document need to be edited via .csv files, which will be
discussed below. Please note, all of the instructions herein are geared
towards PC users.

\subsection{R Markdown and Git}\label{r-markdown-and-git}

The template was created for version control (Git) use in RStudio. R
Markdown integrates plain text, LaTeX, HTML, and R code to create a
reproducible document. The template is an R project (.Rproj), that keeps
all of the files associated with the template organized. More on RStudio
projects
\href{https://support.rstudio.com/hc/en-us/articles/200526207-Using-Projects}{here}.
\href{http://rmarkdown.rstudio.com/}{R Markdown's website} is also full
of resources.

Git is a distributed version control system, that houses files and
repositories on a remote server. When you want to access these
files/repositories, such as the assessment document project, you
checkout a copy from the server, work on it locally (your computer),
commit those changes, and then push your changes to the server. By
committing a snapshot of a file or the repository, Git stores that
version of the file in its memory. Each time you commit changes made to
a file, they are stored in memory. This is a huge advantage if you then
run into an error and need to revert to a previous version of your work.
To revert back, click \emph{Diff} on the Git tab (top right window of
RStudio), and then \emph{Revert}.

Git allows you to easily switch between computers and collaborate with
the other assessment authors. This template is housed on
\href{https://github.com/}{GitHub}, but there are dozens of options
available to host your file. The
\href{http://git-scm.com/book/en/v2}{Pro Git Book} (online and free) is
a great place to learn about Git.

\subsection{Other Git notes}\label{other-git-notes}

\textbf{Collaborating with Git}

There are two options for STAT teams to collaborate on the assessment
document, described below. Remember, the repositories for the two
options below are still public, and anyone can fork the repository to
their own account. If you do not want the public to view your work, you
will have to work through a private repository. If you have a university
affiliation, you can request free private from GitHub. As a warning, on
GitHub, all collaborators on a private repository have owner-level
permissions. Other Git hosts, such as BitBucket, offer free private
repositories if this is a concern.

\emph{Personal repository}

Under this model, one person will ultimately have complete access to the
repository. The best recommendation is to make the STAT team lead, or
whomever will be foremost responsible for document writing, the owner of
the repository. This person will follow the below directions to fork the
StockAssessment\_template to their own account. Anyone who wishes to
contribute to the repository will submit a pull request to the owner to
incorporate changes. A more detailed description on contributing to open
source projects can be found
\href{https://guides.github.com/activities/contributing-to-open-source/}{here}.

\emph{Organization with teams repository}

If you want more than one person to have read/write access to the
repository, you would create an
\href{https://help.github.com/articles/creating-a-new-organization-account/}{Organization}
with
\href{https://help.github.com/articles/permission-levels-for-an-organization-repository/}{Teams}.
Remember, these repositories are still public, and anyone can fork the
repository to their own account.

\section{Getting started}\label{getting-started}

Jennifer (Jenny) Bryan has a top-notch tutorial,
\href{http://happygitwithr.com/}{`Happy Git and GitHub for the useR'} on
the Git basics and connecting GitHub to RStudio. I HIGHLY recommend
reading through her tutorials. I produced some documentation below
(before discovering Jenny's work) and the combination of the two should
get you ready to create R Markdown documents with GitHub and RStudio.

\subsection{Necessary software}\label{necessary-software}

If you have not registered for free \href{https://github.com}{GitHub}
and \href{www.mendeley.com}{Mendeley} or \href{www.zotero.org}{Zotero}
accounts, do so now. We'll cover more on Mendeley later on.

Make sure you have the latest versions of the following programs
installed on your machine:\\

\begin{itemize}[noitemsep,nolistsep,topsep=0pt]
\item \href{https://git-scm.com/book/en/v2/Getting-Started-Installing-Git}{Git} (Contains instructions to install on Linux, Mac, and PC)  
\item \href{https://cran.r-project.org/bin/windows/base/}{R} (latest version is usually a good idea)    
\item \href{https://www.rstudio.com/products/rstudio/download3/}{RStudio} (Note: RStudio Preview will NOT work with the template until a bug is fixed)    
\item PC users: \href{http://miktex.org/}{MikTeX} or \href{https://www.tug.org/texlive/}{TeX Live} (TeX Live requires more work;needed to render a PDF to LaTeX)   
\item Mac users: \href{http://www.tug.org/mactex/}{MacTeX} (needed to render a PDF to LaTeX)   
\item \href{https://www.mendeley.com/}{Mendeley} (or any other citation manager)    
 \end{itemize}

Eric Anderson has directions for the Git and R setup for Mac users
\href{http://eriqande.github.io/rep-res-web/assign/assign-0.html\#assignment-0}{here}.
Mac users, put the line
\texttt{PATH=\$PATH:\textasciitilde{}/Library/TeX/texbin} in your
\texttt{\textasciitilde{}/.bash\_profile} before starting RStudio.

R packages are automatically checked for in the
\texttt{0-Run\_r4ss\_plots.R} script (in the StockAssessment\_template
repository). You can add any other packages you want to load there. You
may get an error and have to install \texttt{rmarkdown} and
\texttt{knitr} manually.

\subsection{Git, GitHub and SSH
authentication}\label{git-github-and-ssh-authentication}

\begin{enumerate}
\def\labelenumi{\arabic{enumi}.}
\item
  You need to tell git who you are. Navigate to Git Bash from your start
  menu, or in RStudio navigate to \emph{Tools \textgreater{} Shell}.
\item
  In Git Bash, set your username and email (if you haven't done this
  previously). The username does not have to be your GitHub username,
  but the email address HAS to be the email you registered with GitHub,
  it can also be your first and last name. In the below commands,
  replace `Melissa Monk' and my email with your name and email:
\end{enumerate}

\begin{quote}
\colorbox{light-gray}{\makebox{git config $--$global user.name 'Melissa Monk'}}\\
\colorbox{light-gray}{\makebox{git config - -global user.email 'melissa.monk@noaa.gov'}}
\end{quote}

\begin{enumerate}
\def\labelenumi{\arabic{enumi}.}
\setcounter{enumi}{2}
\tightlist
\item
  Because these commands return nothing, check to see that git
  registered your information by typing:
\end{enumerate}

\begin{quote}
\colorbox{light-gray}{\makebox{git config --global --list}}
\end{quote}

\begin{enumerate}
\def\labelenumi{\arabic{enumi}.}
\setcounter{enumi}{3}
\item
  If you want to avoid the prompt for your username and password every
  time you push and pull the repository, read the remainder of this
  section. Otherwise, skip it! Getting the SSH-authentication to work
  took me a few tries, so be patient if it doesn't work the first time.
  If the steps below don't work for you, check out
  \href{http://happygitwithr.com/}{Jenny Bryan's tutorial} or this
  \href{http://www.r-bloggers.com/rstudio-pushing-to-github-with-ssh-authentication/}{R-blogger
  post}.
\item
  You can create SSH keys from either RStudio or the shell. First, check
  to see if you have existing keys.
\end{enumerate}

\begin{quote}
Open Git Bash and type:
\end{quote}

\begin{quote}
\colorbox{light-gray}{\makebox{ls -al ~/.ssh}}
\end{quote}

\begin{quote}
If this command tells you .ssh does not exist, you don't have SSH keys.
If the command returns a list of files including id\_rsa.pub
\textgreater{}and id\_rsa, you have a key pair. You can skip to Step 9.
Otherwise, set up the keys via Git Bash or RStudio.
\end{quote}

\subsubsection{Via RStudio (much easier)}\label{via-rstudio-much-easier}

\begin{enumerate}
\def\labelenumi{\arabic{enumi}.}
\setcounter{enumi}{5}
\item
  \begin{enumerate}
  \def\labelenumii{\alph{enumii}.}
  \tightlist
  \item
    In RStudio, in the menu navigate to Tools -\textgreater{} Global
    Options -\textgreater{} Git/SVN. Click \emph{Create RSA Key}. This
    should generate the key. Click \emph{View public key} and copy the
    contents. Go to step 9.
  \end{enumerate}
\end{enumerate}

\subsubsection{Via Git Bash}\label{via-git-bash}

\begin{enumerate}
\def\labelenumi{\arabic{enumi}.}
\setcounter{enumi}{5}
\item
  \begin{enumerate}
  \def\labelenumii{\alph{enumii}.}
  \setcounter{enumii}{1}
  \tightlist
  \item
    In Git Bash, type the following, replacing my email with the email
    associated with your GitHub account:
  \end{enumerate}
\end{enumerate}

\begin{quote}
\colorbox{light-gray}{\makebox{ssh-keygen -t rsa -b 4096 -C "melissa.monk@noaa.gov"}}
\end{quote}

\begin{enumerate}
\def\labelenumi{\arabic{enumi}.}
\setcounter{enumi}{6}
\item
  \begin{enumerate}
  \def\labelenumii{\alph{enumii}.}
  \setcounter{enumii}{1}
  \tightlist
  \item
    Accept the proposed key storage location by pressing Enter. You will
    be prompted to enter an option passphrase to protect the key. Either
    enter a passphrase, or leave this empty and press Enter (my
    preference).
  \end{enumerate}
\item
  \begin{enumerate}
  \def\labelenumii{\alph{enumii}.}
  \setcounter{enumii}{1}
  \tightlist
  \item
    Make sure the ssh-agent is enabled on your machine, by typing
    \emph{eval ``\$(ssh-agent -s)''} into Git Bash. It should return and
    Agent pid.
  \end{enumerate}
\end{enumerate}

\begin{quote}
Add your key to the agent:
\end{quote}

\begin{quote}
\colorbox{light-gray}{\makebox{ssh-add ~/.ssh/id\_rsa}}
\end{quote}

\begin{quote}
If you chose to add a passphrase, you'll be prompted for it here. If
you're successful, the shell will return something like:
\end{quote}

\begin{quote}
\colorbox{light-gray}{\makebox{Identity added: /Users/melissa.monk/.ssh/id\_rsa (/Users/melissa.monk/.ssh/id\_rsa)}}
\end{quote}

\begin{quote}
Go to step 9.
\end{quote}

\begin{center}\rule{0.5\linewidth}{\linethickness}\end{center}

\begin{enumerate}
\def\labelenumi{\arabic{enumi}.}
\setcounter{enumi}{8}
\item
  Navigate to your profile on \href{www.github.com}{GitHub}. From the
  View Profile screen select the \emph{Edit Profile} button in the top
  right. Find \emph{SSH keys} in the left-hand menu, and select
  \emph{New SSH key}. Name the key something like `Laptop key' (you'll
  need a separate key for each computer). Paste the View Public Key
  contents from RStudio into the \emph{Key} box.
\item
  To check that the ssh-authentication works, type the following in Git
  Bash:
\end{enumerate}

\begin{quote}
\colorbox{light-gray}{\makebox{ssh -T git@github.com}}
\end{quote}

\begin{quote}
If it works, you should get something like\\
\emph{Hi your\_username! You've successfully authenticated, but GitHub
does not provide shell access.}
\end{quote}

\begin{enumerate}
\def\labelenumi{\arabic{enumi}.}
\setcounter{enumi}{10}
\tightlist
\item
  Remember to copy the SSH URL from the repository page to clone the
  repository.
\end{enumerate}

\begin{quote}
Change remote.origin.url from HTTPS to HTTP (step may not be necessary)
\end{quote}

\begin{enumerate}
\def\labelenumi{\arabic{enumi}.}
\setcounter{enumi}{11}
\tightlist
\item
  If RStudio is still asking you for a username and password after you
  set up the ssh-authentication, type the following in Git Bash
\end{enumerate}

\begin{quote}
\colorbox{light-gray}{\makebox{git config remote.origin.url}}
\end{quote}

\begin{quote}
\colorbox{light-gray}{\makebox{git@github.com:your\_username/your\_project.git}}
\end{quote}

\subsection{Protecting confidential
information}\label{protecting-confidential-information}

Keep in mind that if you're using a public repository, everything you
push is visible to the world! Even if you realize that you've pushed
confidential information and remove it with your next commit and push to
GitHub, the confidential is \textbf{not actually deleted}!! The only way
to remove the confidential information is to wipe the repository from
GitHub, or go through a laborious process of removing your commit
history.

If you have data files that you want to keep with the repository
locally, you can add it to the .gitignore file. An example of this would
be if you had a text file containing confidential information that you
were then going to use within an R code chunk to create a
non-confidential table or figure for your assessment.

If you're writing R code chunks that use ODBC connections, you can use
the following script for the channel to prompt you for your password:\\
\texttt{channel\textless{}-odbcConnect("database",}\\
\hspace*{0.333em}\hspace*{0.333em}\hspace*{0.333em}\hspace*{0.333em}\hspace*{0.333em}\hspace*{0.333em}\hspace*{0.333em}\hspace*{0.333em}\hspace*{0.333em}\hspace*{0.333em}\hspace*{0.333em}\texttt{uid="username",}\\
\hspace*{0.333em}\hspace*{0.333em}\hspace*{0.333em}\hspace*{0.333em}\hspace*{0.333em}\hspace*{0.333em}\hspace*{0.333em}\hspace*{0.333em}\hspace*{0.333em}\hspace*{0.333em}\hspace*{0.333em}\texttt{pwd=.rs.askForPassword("Enter\ password:"))}

\subsection{Update your fork from
GitHub}\label{update-your-fork-from-github}

If the StockAssessment\_template repository on Melissa's GitHub is
updated, you can update your fork. The directions below work, but they
will leave you ``1 commit head of the master.''

\begin{enumerate}
\def\labelenumi{\arabic{enumi}.}
\tightlist
\item
  Open your fork repository on GitHub.
\item
  Click on \texttt{Pull\ Requests}.
\item
  Click on \texttt{New\ Pull\ Request}. By default, GitHub will compare
  the original with your fork, and there shouldn't be nothing to compare
  if you didn't make any changes.
\item
  Change the \texttt{Base} drop down's so both point to your fork and
  then you'll get a prompt to \texttt{Compare\ across\ repos} (if no
  changes were made in the fork) or click Edit and switch the base
  manually. Now GitHub will compare your fork with the original, and you
  should see all the latest changes.
\item
  Click on \texttt{Create} to create a pull request for this comparison
  and assign a predictable name to your pull request (e.g., Update from
  original).
\item
  Click on Send pull request.
\item
  Scroll down and click \texttt{Merge\ pull\ request} and finally
  Confirm merge (If your fork didn't have any changes, you will be able
  to merge it automatically).
\end{enumerate}

Directions I think will work better are
\href{https://2buntu.com/articles/1459/keeping-your-forked-repo-synced-with-the-upstream-source/}{here},
but I haven't tried them yet!.

\section{Using the template
step-by-step}\label{using-the-template-step-by-step}

\subsection{Forking the template}\label{forking-the-template}

Once you have the necessary software installed and your machine talking
to GitHub, you're ready to fork (make a copy of) the
StockAssessment\_template repository and begin your own assessment
document.

\begin{enumerate}
\def\labelenumi{\arabic{enumi}.}
\tightlist
\item
  Navigate to the
  \href{https://github.com/melmonk/StockAssessment_template}{StockAssessment\_template}
  on my GitHub page, melmonk.
\item
  In the top-right corner of the the StockAssessment\_template
  repository page, click \emph{Fork}.

  \begin{itemize}
  \tightlist
  \item
    You now have a copy of the StockAssessment\_template repository in
    your own account.
  \end{itemize}
\item
  Navigate back to your personal GitHub page.

  \begin{itemize}
  \tightlist
  \item
    StockAssessment\_template should now appear in your list of
    repositories.
  \item
    If you want to rename the repository, go to Settings (in the menus
    below the repository name in blue).
  \end{itemize}
\item
  Click the link to the StockAssessment\_template from your personal
  page.

  \begin{itemize}
  \tightlist
  \item
    On the right-hand side you'll see \emph{HTTPS} clone URL, click the
    copy button, unless you're connecting to the GitHub via \emph{SSH}
    (see the \emph{Github and ssh-authentication section} and give it a
    quick read at this step).
  \item
    This will allow you to pull the repository to your desktop.
  \end{itemize}
\item
  Now open RStudio, navigate to \emph{File -\textgreater{} New Project}.
\item
  Select \emph{Version Control}, then \emph{Git}.
\item
  Paste the URL you just copied on GitHub into the Repository URL box.

  \begin{itemize}
  \tightlist
  \item
    The \emph{Project directory name:} will be autofilled, although you
    can change it.
  \item
    You can also change the \emph{Create project as a subdirectory of}
    box.
  \end{itemize}
\item
  Now click \emph{Create Project}.

  \begin{itemize}
  \tightlist
  \item
    Windows will pop-up asking for your GitHub username and then
    password (unless you have ssh-authentication).
  \item
    The repository will now download.
  \end{itemize}
\item
  Before we start changing files, let's make sure the template knits on
  your machine.

  \begin{itemize}
  \tightlist
  \item
    Navigate to ./RCode/0-Run\_r4ss\_plots.R and run through Section 1.
    You need to do this to produce the r4SS plots which I removed from
    the template repository to speed up download times and prevent
    compiling errors, e.g., the working directory is part of the r4SS
    code and is specific to your machine and inserting plots. Once you
    fork the template and have it on your machine, you can remove the
    `/r4SS/' from the .gitignore file.
  \item
    At the top of the RStudio scripts pane you should see a ball of yarn
    icon that says ``knit.'' Click \textbf{knit to PDF} from the drop
    down menu.
  \item
    If there are no errors, RStudio will execute the R code chunks, and
    convert the document to a PDF and open a preview window with the
    knitted document.
  \item
    If it works, go to step 11. If it doesn't knit, we need to debug
    before you move on.
  \end{itemize}
\item
  You're ready to write your assessment!

  \begin{itemize}
  \tightlist
  \item
    You can manipulate the files/folders on your desktop copy, e.g., csv
    files and figures, and these will show up as changed in the Git tab.
  \item
    To keep the document looking neat, you can hide R code chunks using
    the arrow that appears between the R script line number and the
    start of the R code chunk, just click on it.
  \end{itemize}
\end{enumerate}

\subsection{Species-specific set-up and running the
template}\label{species-specific-set-up-and-running-the-template}

Once you have forked the assessment template and created a new project
in RStudio, you're ready to begin.

\begin{enumerate}
\def\labelenumi{\arabic{enumi}.}
\item
  Navigate to the project folder on your computer and replace the
  contents of the SS folder with the contents of the run(s) you wish to
  use. If you only have one model, copy the SS files to the
  SS/Base\_model1 folder.
\item
  If you don't currently have the project open in RStudio, do that now.
\item
  In RStudio, the project navigation pane is the bottom panel on the
  right, (contains tabs for Files, Plots, Packages, Help, Viewer).
  Navigate to and open ./Rcode/0-Run\_r4ss\_plots.R. You will need to
  change the working directory, number of models, and the file names of
  the control and data files. Do not source this script. Work through
  the end of Section 1 if you only have one model, and continue to
  Section 2 if you have multiple models. Section 3 is to write the r4ss
  output to a .csv file if you'd like. More details in SS and r4ss,
  section \ref{ss-and-r4ss} below.
\item
  In RStudio, open the file: ./Rcode/Preamble.R. Change the all of the
  variables related to your assessment. By default, you can only work in
  the working directory containing the R project, so the working
  directly is automatically set by the session.
\item
  To change the document title, edit the title in the YAML (top of the
  .Rmd document)
\item
  To change the authors and their affiliations, open and edit the
  Titlepage.tex file
\item
  You are now ready to render the first version of the document. At the
  top of the RStudio scripts pane you should see a ball of yarn icon
  that says ``knit.'' Click \textbf{knit to PDF} from the drop down
  menu. If there are no errors, RStudio will execute the R code chunks,
  and convert the document to a PDF and open a preview window with the
  knitted document.
\item
  If you receive an error, read the error message carefully and start
  debugging :) See a list of common errors in Common Errors, section
  \ref{common-errors}.
\item
  Commit and push to GitHub often!
\end{enumerate}

\subsection{Saving, committing, and pushing your
changes}\label{saving-committing-and-pushing-your-changes}

\begin{enumerate}
\def\labelenumi{\arabic{enumi}.}
\item
  Each time you make a change to a file in the repository, that file
  appears in the Git window

  \begin{itemize}
  \tightlist
  \item
    RStudio will automatically save the .Rmd file each time it's knitted
    (or save as usual; Ctrl+S)
  \end{itemize}
\item
  To commit changes, click \emph{Commit} in the Git window.

  \begin{itemize}
  \tightlist
  \item
    The differences between the last commit and current state of the
    file appear
  \end{itemize}
\item
  Stage each file you want to commit by checking the boxes under
  \emph{Stage} in the top-left window

  \begin{itemize}
  \tightlist
  \item
    Once checked, the status will change to ``M,'' indicating the file
    has been modified
  \item
    If you're deleting a file, the status will change to ``D,''
    indicating the file has been deleted
  \item
    If you're adding a file, the status will change to ``A,'' indicating
    the file has been added
  \item
    Staged files are now ready to be committed
  \end{itemize}
\item
  Before you can commit files, you have to write a short (at least one
  character) message in the top-right \emph{Commit message} box
\item
  You can now click \emph{Commit}
\item
  If you're finished with your session or just want to send your changes
  back to GitHub, now's the time to push them
\item
  Click the green Up arrow in the Git tab

  \begin{itemize}
  \tightlist
  \item
    This prompts you for your GitHub username and password and then
    proceeds to push your changes
  \item
    See the section below on ssh-authentication if you want to avoid
    entering your username and password each time you push or pull the
    repository
  \item
    When you're done for the day navigate to File -\textgreater{} Close
    project
  \item
    To start working on the project again, open RStudio, navigate to
    File -\textgreater{} Open Project, find your file, open it, and pull
    (blue down arrow from the Git tab) any updates from the repository
    to your local machine.
  \end{itemize}
\end{enumerate}

Remember, you can always go back to an older version of the document you
committed. Sometimes you'll get an error where it's easier to wipe the
project from your desktop and pull a new version from GitHub than debug.

If you're working in a repository that has multiple owners, the most
common error is that the other person has pushed changes to GitHub while
you're still working. When you try and push your changes, you're going
to get a Merge error. This can be a headache so try and avoid working
simultaneously. I recommend using Google Docs to work on text, and
having one person take responsibility of the document.

\subsection{SS and r4ss}\label{ss-and-r4ss}

The StockAssessment\_template repository contains both your SS and r4ss
output for the model(s) to include in the document.

The SS folder contains four subfolders, one for each of up to three
model outputs, and a linebreak\_files folder that will contain the
formatted SS input files, i.e., control, forecast, etc. When you're
ready to replace the China rockfish model with your own, open the
appropriate Base\_model\# folder and copy your SS model files here. Do
this for the other Base\_model\# folders if you have multiple models. If
you have only one model, you can empty the contents of the subsequent
folders.

Note: you can rename ``Base\_model1'' to something more intuitive. If
you do, you will need to change the folder name in the
``\texttt{Rcode/0-Run\_r4ss\_plots.R}'' script.

Run the script \texttt{0-Run\_r4ss\_plots.R}. If you have only one
model, you can skip the plot comparisons section. You can also add plot
modifications in this script, e.g., extending plot margins for a plot
group. The SS\_output object(s) will be saved within the workspace
\texttt{SS\_output.RData} in the r4ss subfolder. Within the r4ss
subfolder, model specific plots are saved in a \texttt{plots\_mod\#}
folder (i.e., \texttt{plots\_mod1}) and model comparison plots are saved
in \texttt{plots\_compre}. The r4ss subfolder also contains placeholder
folders for forecast plots (\texttt{plots\_forecast}), profile
likelihood plots (\texttt{plots\_profiles}), retrospective analysis
plots (\texttt{plots\_retros}), and sensitivity analysis plots
(\texttt{plots\_sensitivity}).

\section{The document, section by
section}\label{the-document-section-by-section}

All files used within the project need to be in the same parent folder.
Be careful pasting text from Word, Google Docs, Textpad, etc. into R.
Symbols such as a hyphens, quotes, apostrophes may not copy
``correctly'' and will either not show up when the document is knitted,
or will be knitted as the incorrect symbol.

\subsection{A list of folders and files in the Assessment template
project}\label{a-list-of-folders-and-files-in-the-assessment-template-project}

This list is ever-evolving, but will give you a sense of what's included
in the repository, in order as they appear by default in the RStudio
Files window (bottom right pane).

\begin{itemize}
\item
  \textbf{.gitignore}
  \textcolor{magenta}{Contains the names of files and file types to ignore when pushing and pulling}
\item
  \textbf{.Rhistory}
  \textcolor{magenta}{Contains the R session history, included in .gitignore}
\item
  \textbf{accessibility-meta.sty}
  \textcolor{magenta}{Style to create an ADA accessible document - currently doesn't work, but someone can try and figure it out!}
\item
  \textbf{Assessment\_template.* (.pdf, .Rmd, .proj, .tex)}
  \textcolor{magenta}{The .Rmd file is the main document for the Assessment template.  The .proj file is the RStudio file directory that houses the entire Assessment Template project, and is the file you open into RStudio.  The TeX and PDF files are generated when you knit the document.}
\item
  \textbf{Assessment\_template\_files folder}
  \textcolor{magenta}{This folder contains files automatically generated and are included in the gitignore file list}
\item
  \textbf{BibFile.bib}
  \textcolor{magenta}{This is the bibliography file, which I generate in Mendeley}
\item
  \textbf{CJFAS.csl}
  \textcolor{magenta}{This is the citation style file for the bibliography.  This particular citation style file creates a references section following the format of the Canadian Journal of Fisheries and Aquatic Sciences}
\item
  \textbf{cover\_photo.png}
  \textcolor{magenta}{If you want to include a picture of your fish species on the cover, replace this picture with yours, and name it cover\_photo.}
\item
  \textbf{Default\_template\_modified.tex}
  \textcolor{magenta}{The default pandoc template had a text rendering issue for text with all capital letters when viewed in Adobe.  This modified version of the template comments out the lines that caused the issue (hopefully).}
\item
  \textbf{Example\_tables\_figures folder}
  \textcolor{magenta}{This folder contains examples of figures and tables and how to create them.  Look here if there's a table or figure from the 2015 China rockfish assessment that you want to mimic, but can't figure out how.}
\item
  \textbf{Figures folder}
  \textcolor{magenta}{This is where all of the user-created plots are stored (NOT r4ss generated plots).}
\item
  \textbf{header.tex}
  \textcolor{magenta}{This file contains a list of the LaTeX packages to use in the document.}
\item
  \textbf{r4ss folder}
  \textcolor{magenta}{This is where all of the r4ss generated plots are stored.}
\item
  \textbf{Rcode folder}
  \textcolor{magenta}{This folder houses all of the larger R code scripts needed for the document.}
\item
  \textbf{ReadMe folder}
  \textcolor{magenta}{The .Rmd file is the source for the ReadMe file for the Assessment Template.  The TeX file is generated when the document is knit, and the PDF is for the user to read!}
\item
  \textbf{SS folder}
  \textcolor{magenta}{This folder houses all of the Stock Synthesis files for each base model.}
\item
  \textbf{SS\_file\_appendices.Rmd}
  \textcolor{magenta}{This is the child .Rmd file that creates the SS file Appendices.  You shouldn't need to edit this file.}
\item
  \textbf{Test\_figures\_tables.Rmd}
  \textcolor{magenta}{This .Rmd file is where you can test R code chunks figures and tables before adding them to the main Assessment\_template .Rmd file, that you render to Test\_figures\_tables.pdf.  It has to be in the main folder to work.  This will save you a lot of time debugging!}
\item
  \textbf{Titlepage.tex}
  \textcolor{magenta}{This LaTeX file generates the title page and should be edited to include the correct authors.}
\item
  \textbf{txt\_files folder }
  \textcolor{magenta}{This folder houses all of the text (.txt or .csv) files used to generate tables.}
\end{itemize}

\subsection{The .gitignore file}\label{the-.gitignore-file}

RStudio automatically creates the \texttt{.gitignore} file for a new
project, and the Assessment\_template document contains one in the root
directory. This file lists all of the files/file types you want Git to
ignore when you commit and push files. For instance, you don't need to
push/pull the Assessment\_template repository. If you add files to
.gitignore, you have to delete and commit them as deleted before they
will be ignored.

Below is the .gitignore file for Assessment\_template. It ignores the
Rproject user file, the Rhistory files, the files associated with
knitting the document, as well as any files containing `unnamed-chunk'
(files created by LaTeX every time you knit). More info on ignoring
files \href{http://git-scm.com/docs/gitignore}{here}.

\begin{Verbatim}[frame=single]
===  
.Rproj.user 
===   
.Rhistory  
===   
/Assessment_template.pdf
/Assessment_template.html
Assessment_template.tex
Test_figures_tables.tex
*.docx

**/list_of_dataframes.csv
**/mod_structure.csv
===  Latex temp files  
*unnamed-chunk*
\end{Verbatim}

\newpage

\subsection{The YAML}\label{the-yaml}

The YAML contains all of the document front-matter and must be the first
set of code in any .Rmd file. You cannot add comments to the YAML. Each
YAML element used in the assessment template is described below. The
document is currently only authored to be knit to a pdf file.\\
\texttt{-\/-\/-}
\textcolor{magenta}{YAML begins with a line of three hyphens, no spaces}\\
title: ``''
\textcolor{magenta}{Provide the title of the document in quotes}\\
author: `'
\textcolor{magenta}{Leave blank, authors are defined in Titlepage.tex}\\
date:''
\textcolor{magenta}{Leave blank, date is defined in Titlepage.tex}\\
output: \textcolor{magenta}{Begin defining output variables}\\
\hspace*{0.333em}\hspace*{0.333em} pdf\_document:
\textcolor{magenta}{Begin defining pdf output variables}\\
\hspace*{0.333em}\hspace*{0.333em}\hspace*{0.333em}\hspace*{0.333em}
fig\_caption: yes
\textcolor{magenta}{Should figures be captioned? yes or no}\\
\hspace*{0.333em}\hspace*{0.333em}\hspace*{0.333em}\hspace*{0.333em}
highlight: haddock
\textcolor{magenta}{Color scheme for highlighting R code; options below}\\
\hspace*{0.333em}\hspace*{0.333em}\hspace*{0.333em}\hspace*{0.333em}
includes: \textcolor{magenta}{Define external documents to include}\\
\hspace*{0.333em}\hspace*{0.333em}\hspace*{0.333em}\hspace*{0.333em}\hspace*{0.333em}\hspace*{0.333em}\hspace*{0.333em}
before\_body: Titlepage.tex
\textcolor{magenta}{Include Titlepage.tex (title page first)}\\
\hspace*{0.333em}\hspace*{0.333em}\hspace*{0.333em}\hspace*{0.333em}\hspace*{0.333em}\hspace*{0.333em}\hspace*{0.333em}
in\_header: header.tex
\textcolor{magenta}{Include header.tex (all necessary LaTeX packages)}\\
\hspace*{0.333em}\hspace*{0.333em}\hspace*{0.333em}\hspace*{0.333em}
keep\_tex: yes
\textcolor{magenta}{Keep intermediate TeX output? yes or no}\\
\hspace*{0.333em}\hspace*{0.333em}\hspace*{0.333em}\hspace*{0.333em}
latex\_engine: xelatex
\textcolor{magenta}{Define the latex engine (sometimes matters)} ~~~~
template: Default\_template\_modified.tex
\textcolor{magenta}{Template comments out lmodern package} ~~~~
number\_sections: yes
\textcolor{magenta}{Number the document sections? yes or no}\\
\hspace*{0.333em}\hspace*{0.333em}\hspace*{0.333em}\hspace*{0.333em}
toc: yes \textcolor{magenta}{Include a table of contents? yes or no}\\
\hspace*{0.333em}\hspace*{0.333em}\hspace*{0.333em}\hspace*{0.333em}
toc\_depth: 4
\textcolor{magenta}{Number of subheadings to include in the table of contents}\\
\hspace*{0.333em}\hspace*{0.333em} html\_document:
\textcolor{magenta}{Begin defining HTML output variables}\\
\hspace*{0.333em}\hspace*{0.333em}\hspace*{0.333em}\hspace*{0.333em}
toc: yes \textcolor{magenta}{Include a table of contents? yes or no}\\
\hspace*{0.333em}\hspace*{0.333em} word\_document: default
\textcolor{magenta}{Begin defining Word document output variables}\\
documentclass: article \textcolor{magenta}{LaTeX document class}\\
fontsize: 12pt \textcolor{magenta}{Default font size}\\
geometry: margin=1in \textcolor{magenta}{Page margin size}\\
csl: CJAFS.csl \textcolor{magenta}{Bibliography style}\\
bibliography: BibFile.bib \textcolor{magenta}{Bibliography file name}\\
link-citations: yes \textcolor{magenta}{Adds hyperlinks to citations}
\texttt{-\/-\/-}
\textcolor{magenta}{YAML ends with a line of three hyphens, no spaces}

Notes:

\begin{itemize}
\item
  Keeping the TeX file can help with debugging.

  \begin{itemize}
  \tightlist
  \item
    The line number for a given error can refer to .Rmd, .tex file, or
    be completely random
  \end{itemize}
\item
  Options for the R script highlighting include: default, tango,
  pygments, kate, monochrome, espresso, zenburn, haddock and textmate.
  Play around with them to see which color you like best.
\item
  The HTML and Word document settings are currently dummy settings just
  so you can knit to these. Future work can be done to knit the document
  to these formats.
\item
  I've included the modified default pandoc template that comments out
  the lmodern package and other associated packages. These packages
  cause some strange font rendering of acronyms (or other words in all
  capital letters) when viewed in Adobe products.
\end{itemize}

\subsection{The Meat and Bones}\label{the-meat-and-bones}

The Assessment Template contains most (if not all) of the headers in the
Terms of Reference. I have left bits and pieces in the document that
likely apply to all assessments, e.g., the citation for the Hamel prior
in the Priors section. The following sections will provide details on
each section of the template.

\subsection{Executive Summary}\label{executive-summary}

The Executive Summary is basically written and calls the r4SS output and
csv files you provide. You will have to edit the text (.csv) files, such
as the catch histories and landings by fleet, and decision tables. For
all of these, you need to edit the text file and possibly the R code (in
Rcode/R\_exec\_summary\_figs\_tables.R) depending on the table/figure
structures.

The following Executive Summary tables are associated with .csv files,
which will need to be replaced with your data. This may also require
editing the column alignment options if you have a different number of
columns than in the default template. This is where you'll want to test
your tables and figures in the Test\_figures\_tables.Rmd file, by
copying the code to the main documents.

\begin{itemize}
\tightlist
\item
  Table a. Recent landings by fleet.

  \begin{itemize}
  \tightlist
  \item
    \texttt{Exec\_catch\_summary.csv}
  \end{itemize}
\item
  Table n. Recent trend in total catch\ldots{}relative to management
  guidelines.

  \begin{itemize}
  \tightlist
  \item
    \texttt{Exec\_mngmt-performance.csv}
  \end{itemize}
\item
  Tables p-r. Decision table(s)

  \begin{itemize}
  \tightlist
  \item
    \texttt{DecisionTable\_mod1.csv} (and if needed
    \texttt{DecisionTable\_mod2.csv} and
    \texttt{DecisionTable\_mod3.csv})
  \end{itemize}
\item
  Table s. Base case results summary. Note: This table is a mix of a
  .csv file and r4SS output.

  \begin{itemize}
  \tightlist
  \item
    \texttt{Exec\_basemodel\_summary.csv} and r4ss output. the .csv file
    contains the harvest guidelines. All other values are pulled from
    the r4ss output.
  \end{itemize}
\end{itemize}

\subsection{Appendices: SS input files}\label{appendices-ss-input-files}

The script in \texttt{Run\_SS\_input\_linebreaks.R} contains the
function and then commands to edit the SS input files for printing in
Appendices A-D, where Appendix A: SS data file, Appendix B: SS control
file, Appendix C: SS starter file, and Appendix D: SS forecast file. The
script for the appendices is contained in a child .Rmd file,
\texttt{Appendices.Rmd}. The Appendices A-D are appended to the document
via the following R code chunk:

\begin{Verbatim}[frame=single]
<!-- ```{r child="Appendices.rmd"}  # '<!--' opens an HTML comment
         ``` -->                    # '-->'  ends an HTML comment
\end{Verbatim}

The SS appendix files are currently commented out in the
Assessment\_template.Rmd file, as well as in the above box, using HTML
comment syntax. They are commented to save on runtime and reduce the
document size while editing. Remove the HTML comment syntax to include
the SS appendices.

\subsection{References section}\label{references-section}

If you have a citation manager you're committed to that will create a
.bib file, you can skip this section.

\textbf{Create the References section with Mendeley}

If you have not already downloaded
\href{https://www.mendeley.com/}{Mendeley} and created a free account,
do so now.

R Markdown can read a number of bibliography styles, see
\href{http://rmarkdown.rstudio.com/authoring_bibliographies_and_citations.html}{Bibliographies
section}. I'm providing directions for creating a .bib file in Mendeley.
Using Mendeley is not required, but it's free and a great citation
manager. The biggest downfall with Mendeley is that you cannot include
italics in title names. I provide a (somewhat clunky) work-around below.
One other caveat of R Markdown is that the References section is
automatically placed at the end of the document, which means after the
table, figures, and appendices.

\begin{enumerate}
\def\labelenumi{\arabic{enumi}.}
\tightlist
\item
  Download Mendeley Desktop
  \href{https://www.mendeley.com/download-mendeley-desktop/}{here} and
  also install the Web Importer (for use with Google Scholar, or a
  journal's website). If you don't already have a free account, create
  one now.
\item
  Create a group on Mendeley for collaboration, e.g., China Rockfish
  Assessment 2015, if you want to collaborate on the references section,
  i.e., allow your co-authors to add citations.

  \begin{itemize}
  \tightlist
  \item
    \emph{File \textgreater{} New Group}
  \item
    The setup should be self-explanatory.
  \end{itemize}
\item
  Either input a reference manually or import it to Mendeley from Google
  Scholar.

  \begin{itemize}
  \tightlist
  \item
    Ensure all the pieces, e.g., page numbers, are imported. If not,
    enter them manually.
  \item
    Check this for each reference, or else you will be re-doing the
    search for all of your literature at the last minute.
  \end{itemize}
\item
  Add references to the group folder in Mendeley Desktop.

  \begin{itemize}
  \tightlist
  \item
    The reference will also be added to your main library.
  \end{itemize}
\item
  Make sure the Citation Key field is not blank and matches the key you
  want to reference it as in the R Markdown document.

  \begin{itemize}
  \tightlist
  \item
    The citation key is used to cite documents in the assessment.
  \item
    Best practice: use the first author's last name and the year of
    publication, e.g., Monk2015.
  \end{itemize}
\item
  To update the .bib file, Go to Documents tab in the group folder,
  select all, and go to \emph{File \textgreater{} Export}.

  \begin{itemize}
  \tightlist
  \item
    Export the files as a .bib file.
  \item
    Save and overwrite the BibFile.bib file in your version-controlled
    working folder.
  \end{itemize}
\item
  Make sure you include the new .bib file when you push your changes to
  GitHub.
\end{enumerate}

The .bib file is rendered via LaTeX, so you can enclose a scientific
name with \texttt{\textbackslash{}emph\{\}} in the Mendeley citation,
e.g., \texttt{\textbackslash{}emph\{Sebastes\ nebulosus\}}, to produce
\emph{Sebastes nebulosus}. To make sure these render properly, go to
Tools -\textgreater{} Options -\textgreater{} BibTeX and uncheck the box
that says ``escape LaTeX special characters.''

To add a reference to the document type {[}@CitationKey{]}, which will
include the reference in parentheses. To include the reference as the
year only, type {[}-@CitationKey{]}. If you include a year only
citation, remember to manually type in the author part of the citation.

The 2015 China rockfish assessment bibliography collection is public in
Mendeley. You can find it in Mendeley by searching for the group ``China
Rockfish Assessment 2015.'' Once you join the group, you can create a
new group and drag all of the references you want from the China
Rockfish assessment to your new group.

\newpage

\subsection{Before you publish}\label{before-you-publish}

There are a few LaTeX packages turned on in the default template on
GitHub. These are controlled in the header.tex file. To turn a feature
off, comment out the package with a \%.

\begin{enumerate}
\def\labelenumi{\arabic{enumi}.}
\item
  Package lineno allows for line numbers throughout the document. This
  was helpful for reviewers during the STAR panel and editing post-STAR
  panel.
\item
  Package showlabels prints the section, figure and table labels. This
  is helpful if you're trying to remember which figure/table you're
  cross-referencing in the text.
\item
  Package draftwatermark places a water mark on the pages.
\end{enumerate}

\section{Creating Tables}\label{creating-tables}

Tables are generated within R code chunks using the R package xtable.
The
\href{https://cran.r-project.org/web/packages/xtable/index.html}{xtable
vignettes} are extremely useful. I recommend starting there if you have
a question.

Tables in the Assessment\_template are generated from R output,
including r4ss, or a .txt/.csv file (located in the \texttt{txt\_files}
subfolder). You'll create tables inside R code chunks using the xtable
package. I highly recommend using the Test\_figures\_tables.Rmd document
to test run any new tables you want to add to the document.

Create a table using these general steps:

\begin{enumerate}
\def\labelenumi{\arabic{enumi}.}
\item
  You can either read in a .csv file or manipulate data from the r4ss
  output. Either way, the dataframe should resemble the same format (row
  and columns) that you want in the table. Start an R code chunk and
  read in the data.
\item
  Edit the column names. R will likely remove spaces if done before data
  manipulation is complete.
\item
  Create the table using the xtable command.

  \begin{itemize}
  \tightlist
  \item
    \texttt{xtable(dataframe\_name,\ caption=c("Table\ caption"),\ label=\textquotesingle{}tab:table\_label})
  \item
    The label allows you reference the table in the document, ex.
    \texttt{\textbackslash{}ref\{tab:table\_label\}} and will
    automatically number each table.
  \item
    I find it good practice to precede a table label with `tab' so you
    can easily recognize the reference (e.g., tab:Exec\_catch)
  \end{itemize}
\item
  Adjust the column alignment

  \begin{itemize}
  \tightlist
  \item
    \texttt{align(table\_name)\ =\ c(\textquotesingle{}l\textquotesingle{},\textquotesingle{}l\textquotesingle{},\textquotesingle{}\textgreater{}\{\textbackslash{}\textbackslash{}centering\}p\{1in\}\textquotesingle{})}
  \item
    You must have one dummy column alignment parameter for row.names, if
    you are not printing row names\\
  \item
    Common alignment options

    \begin{itemize}
    \tightlist
    \item
      \texttt{\textquotesingle{}l\textquotesingle{}},
      \texttt{\textquotesingle{}c\textquotesingle{}}, or
      \texttt{\textquotesingle{}r\textquotesingle{}} for left, center or
      right alignment (these options do not adjust column width)
    \item
      \texttt{\textquotesingle{}\textgreater{}\{\textbackslash{}\textbackslash{}centering\}p\{1in\}\textquotesingle{}}
      for center alignment where you assign the column width, 1 inch in
      this case
    \item
      \texttt{\textquotesingle{}\textgreater{}\{\textbackslash{}\textbackslash{}raggedright\}p\{1in\}\textquotesingle{}}
      for left alignment where you assign the column width, 1 inch in
      this case
    \item
      \texttt{\textquotesingle{}\textgreater{}\{\textbackslash{}\textbackslash{}raggedleft\}p\{1in\}\textquotesingle{}}
      for right alignment where you assign the column width, 1 inch in
      this case
    \end{itemize}
  \end{itemize}
\item
  Print the table

  \begin{itemize}
  \tightlist
  \item
    print(table\_name)
  \item
    Common print options include

    \begin{itemize}
    \tightlist
    \item
      include.rownames=FALSE (don't include rownames as a column)
    \item
      caption.placement = ``top'' (place caption above the table)
    \item
      sanitize.text.function = function(x)\{x\} (you'll see this where I
      include LaTeX syntax in the table - however, including this when
      not needed produces an error)
    \item
      hline.after = c(-1,0,6,12) (include extra horizonal lines in the
      table after the line specified)
    \item
      scalebox = .6 (scales the table if it doesn't fit on the page,
      value of .6 is 60\% original size)
    \item
      floating.environment=``sidewaystable'' (creates a table in
      landscape mode, but you cannot move the page number to another
      side)
    \item
      tabular.environment=``longtable'' (creates a table spanning
      multiple pages)
    \item
      add.to.row=addtorow (include if you're adding rows to the top of
      the table, see Spanning multiple columns below)
    \item
      size = ``small'' (or any other recognized default LaTeX font size)
    \end{itemize}
  \end{itemize}
\end{enumerate}

To create a table that is both in landscape mode and spans multiple
pages, you create the table as a longtable, and rotate the page. See the
model parameters table for an example.

\subsection{Including special characters in
tables}\label{including-special-characters-in-tables}

Adding special characters or bold/italic font to cells in a table varies
slightly depending on if the table is read in from a text file or if the
table is created directly from R output.

\subsubsection{Table content from a text file (.txt or
.csv)}\label{table-content-from-a-text-file-.txt-or-.csv}

You will have to manually add in the LaTeX syntax for bold or italic
font into the text file. For example, if you want the year 2005 bolded
the cell will read
\texttt{\textbackslash{}\textbackslash{}textbf\{2005\}}. Or for example
you want to bold Total Catch OY, you type
\texttt{\textbackslash{}\textbackslash{}textbf\{Total\ Catch\ OY\}} into
the cell. When you create the table with xtables, you must include the
option, \texttt{sanitize.text.function\ =\ function(x)\{x\}}. Example
taken from Table n of the 2015 China rockfish assessment, created using
the ./txt\_files/Exec\_mngmt\_performance.csv file.

\subsubsection{Table content from R code
chunks}\label{table-content-from-r-code-chunks}

LaTeX syntax can be included in R code or R code chunks. You must
precede each LaTeX command with two backslashes instead of one. The
following will bold and italicize the text:
\texttt{\textbackslash{}\textbackslash{}textbf\{\textbackslash{}\textbackslash{}textit\{Reference\ points\ based\ on\ SPR\ proxy\ for\ MSY\}\}}.
For an example, see the Reference Points table code in
./RCode/R\_exec\_summary\_figs\_tables.R file, which is Executive
Summary Table m.

If you include a percent sign \% in the R code or R code chunk, it needs
to be preceded by two backslashes,
\texttt{\textbackslash{}\textbackslash{}\%}. However, this is not
necessary in the main text of the R Markdown document.

You can use inline math mode just as you would in the main text. For
example, \texttt{\$SPR\_\{B40\textbackslash{}\textbackslash{}\%\}\$}
produces \(SPR_{B40\%}\) within an R code chunk. Note that if you are
using a \% sign in math mode in the main text, which is LaTeX, it must
be preceded with one backslash.

\subsection{Spanning multiple columns}\label{spanning-multiple-columns}

Sometimes you may want a column header to to span multiple columns,
equivalent to the merge columns function in Excel. This can be done
within xtable and may take some playing around with. See Executive
Summary Table p, the Decision Table, for an example. The ``States of
Nature'' column header spans multiple columns. The following code is
specific to this table, but I'll explain each line, numbered in the box
below. For this to be included in the table, you will need to include
\verb|add.to.row=addtorow| in the print() table command.

\begin{Verbatim}[frame=single]
  1. addtorow <- list()
  2. addtorow$pos <- list()
  3. addtorow$pos[[1]] <- -1
     addtorow$pos[[2]] <- -1
  4. addtorow$command <- 
           c('\\multicolumn{3}{c}{} 
            & \\multicolumn{2}{c}{} 
            & \\multicolumn{2}{c}{\\textbf{States of nature}} 
            & \\multicolumn{2}{c}{} \\\\\n', 
           
            '\\multicolumn{3}{c}{} 
           & \\multicolumn{2}{c}{Low M 0.05} 
           & \\multicolumn{2}{c}{Base M 0.07} 
           & \\multicolumn{2}{c}{High M 0.09} \\\\\n')
        
\end{Verbatim}

\begin{enumerate}
\def\labelenumi{\arabic{enumi}.}
\tightlist
\item
  The \verb|addtorow <- list()| creates the addtorow variable as a list,
  which should have two components, pos and command.
\item
  The \verb|addtorow$pos <- list()| turns the pos component into a list
  which will contain the positions of the the rows you're adding
\item
  The \verb|addtorow$pos[[1]] <- -1| and \verb|addtorow$pos[[2]] <- -1|
  set both the rows we're adding to appear before the column names of
  the table.
\item
  The \verb|addtorow$command| creates the list of column headers and
  this is where we can insert \verb|\\multicolumn|. All of the
  information for a single row is in single quotes ending with
  \verb|\\\\\n|. We're adding two rows to the dataset, so we have two
  sets of row commands. There are five backslashes preceding the `n'
  because backslashes get lost in translation, just like we're using two
  backslashes again to call multicolumn.
\end{enumerate}

The first thing to note is the decision table has nine columns. The
value in the first set of curly brackets after multicolumn (3 in this
first row, \verb|\\multicolumn{3}|) gives the number of columns to span
and must add up to the number of columns in the table. So here, 3+2+2+2
= 9 (you don't need to worry about an extra column for row names). The
second set of curly brackets gives the centering for the text, here all
of which are `c' for center. The text for the multicolumn goes in the
third set of curly brackets, and can also be left blank.

\subsection{Horizontal and vertical
lines}\label{horizontal-and-vertical-lines}

You can include extra horizontal lines between any two rows of a table
by including hline.after=c(), wherein you list the table rows after
which you want to insert a horizontal line. Row number -1 will place a
horizontal line above the header, 0 will print a line below the header
and any other number, say 10, will print a horizontal line after row 10.
See Executive Summary Table p, the Decision Table, for an example.

Vertical lines are added in the alignment command for xtable. Vertical
lines are inserted as vertical bars, \textbar{}, and in the example
below, to the right of a column. This example is also taken from the
Executive Summary Decision table, Table p.

\begin{Verbatim}[frame=single]

align(decision_mod2.table) = c('l','l|','c','c|',
                               '>{\\centering}p{.7in}', 'c|',
                               '>{\\centering}p{.7in}','c|',
                               '>{\\centering}p{.7in}','c') 
                               
\end{Verbatim}

\subsection{Shading table cells}\label{shading-table-cells}

TOADS

\section{Creating/Inserting Figures}\label{creatinginserting-figures}

Figure files (pre-existing image files) are incorporated via R Markdown
syntax (e.g., r4ss figures) or R code chunks (e.g., create a figure
within the template using your favorite graphics package). You may not
introduce linebreaks in the R Markdown figure syntax.

The following code will add figure.png from the plots folder to the
document.

\begin{Verbatim}[frame=single]
![Figure caption. \label{fig:figure_label}](plots/figure.png)
\end{Verbatim}

For the standard r4ss plots, remember to change the plot directory,
e.g., `r4ss/plots\_mod3' to access plots for model \#3. I like to
precede the figure label with `fig:' in case there are figures and plots
referencing the same data. I also find it helpful to label r4ss figures
with the model number, e.g., Mod3\_, followed by the figure name (see
below for an example).

\begin{Verbatim}[frame=single]
![Figure caption. \label{fig:Mod3_comp_lendat_flt5mkt2}](r4ss/plots_mod3/
comp_lendat_flt5mkt2.png)
\end{Verbatim}

To create a figure within R code chunks, I find it helpful to actually
use two R code chunks, one for data manipulation and a second to plot
the figure. This is because you want to set include=FALSE for the chunk
manipulating the data, and include=TRUE (default value) for the chunk
plotting the figure, and also so we can include the figure caption. An
example is below. Notice too, you need to only change the R code chunk
options if they differ from the document's global options (described in
the R Code Chunks section).

\begin{Verbatim}[frame=single]
```{r, include=FALSE}    #include is the only option we need to change
     CA_rec_remov = CA_rec_removal[,1:5]
     colnames(CA_rec_remov) = c('Year','South PC','South PR',
                                'North PC','North PR')
     CA_rec_remov1 = melt(CA_rec_remov,id='Year')
     colnames(CA_rec_remov1) = c('Year','Fleet','Removals')
```
```{r,fig.cap="Removals (mt) from the California recreational party/charter 
   and private sectors, north and south of $40^\\circ 10^\\prime$. 
\\label{fig:CA_rec_removal}"}
      ggplot(CA_rec_remov1, aes(x=Year, y=Removals,fill=Fleet)) +
      geom_area(position='stack') + 
      scale_fill_manual(values = 
          c( 'lightsteelblue3','coral',"aquamarine2","mediumpurple")) + 
      scale_x_continuous(breaks=seq(1928,2014,10)) + 
      ylab("Removals (mt)")
``
\end{Verbatim}

\section{General topics}\label{general-topics}

\subsection{Syntax (R Markdown and
LaTeX)}\label{syntax-r-markdown-and-latex}

The syntax used throughout will depend on the output file type, PDF or
HTML. If you only want to knit to a pdf, you can use either the R
Markdown or LaTeX syntax in the examples below. However, if you want to
be able to knit to HTML you're better off using the R Markdown syntax,
since HTML will not render LaTeX. See the section on
\href{http://rmarkdown.rstudio.com/authoring_pandoc_markdown.html}{pandoc
markdown} for more on syntax.

Knitting to a Word document is unstable and will result in
strange/missing output.

\subsection{Paragraphs}\label{paragraphs}

To separate paragraphs, you must leave a blank line between paragraphs.
You can create hard line breaks (without a blank line between
paragraphs) by either leaving two or more spaces at the end of the last
paragraph or a backslash between the last paragraph and the new line.

The width of the text will depend on how wide your RStudio window is,
unless you choose to use carriage returns to keep the text as a maximum
width in the viewing window. R Markdown ignores carriage returns, and
will not start a new paragraph without a blank line.

\subsection{Spell checking}\label{spell-checking}

Do not completely rely on RStudio's spell checker, which doesn't even
include RStudio as a word. Use the spell checker, but edit the document
yourself as well. There is not autocorrect function, as in Word.

\subsection{LaTeX}\label{latex}

If you're looking for a specific LaTeX symbol or command, it's easiest
just to Google it.

You can use LaTeX commands throughout the document, such as
`\textbackslash{}newpage' and `\textbackslash{}FloatBarrier,' which are
helpful for inserting a page break and keeping figures from being
rearranged. I've also found that it's good to insert the float barrier
command after every three figures or so, to prevent a runtime error.

\subsection{Fonts and font size}\label{fonts-and-font-size}

You cannot control both the font and the font size in R Markdown. The
font size specified in the YAML (see YAML section) is set at 12pt.

You can specify italics and bold fonts using either LaTeX or R Markdown
syntax. If you only want to knit to a PDF, my preference is to stick to
the LaTeX syntax, as you'll see throughout the document, but that's
personal opinion.

\begin{Verbatim}[frame=single]
*Italics*  
R Markdown: *word* or  _word_  
LaTeX: \emph{word}


**Bold**     
R Markdown: **word** or __word__ 
LaTeX:  \textbf{word} 

\end{Verbatim}

\subsection{Section headers}\label{section-headers}

Numbered headers in R Markdown are as follows:

\begin{Verbatim}[frame=single]
#Header 1  
##Header 2 
###Header 3
\end{Verbatim}

To create a header without a number, e.g., for the Executive Summary
sections, follow the header with \{-\}, ex. \texttt{\#Header\{-\}}.

Un-numbered subsection headers that you don't want to appear in the
table of contents can be created by starting a section with a bold or
italics header.

\subsection{Numbering (pages, tables,
figures)}\label{numbering-pages-tables-figures}

The table of contents will automatically be numbered using lower case
roman numerals. Arabic numbering of pages begins with the Executive
Summary. Tables and figures in the Executive summary are lowercase
alphabetic. This is defined using the following LaTeX script:

\texttt{\textbackslash{}pagenumbering\{arabic\}}
\textcolor{magenta}{Defines page numbering as Arabic numbers}\\
\texttt{\textbackslash{}setcounter\{page\}\{1\}}
\textcolor{magenta}{Sets the first page number to 1}\\
\texttt{\textbackslash{}renewcommand\{\textbackslash{}thefigure\}\{\textbackslash{}alph\{figure\}\}}
\textcolor{magenta}{Defines figure labels as alphabetic}\\
\texttt{\textbackslash{}renewcommand\{\textbackslash{}thetable\}\{\textbackslash{}alph\{table\}\}}
\textcolor{magenta}{Defines table labels as alphabetic}

\vspace{.5cm}

Immediately preceding the Introduction section, the labels for figures
and tables are reset:

\texttt{\textbackslash{}renewcommand\{\textbackslash{}thefigure\}\{\textbackslash{}arabic\{figure\}\}}
\textcolor{magenta}{Defines figure labels as Arabic numbers}\\
\texttt{\textbackslash{}renewcommand\{\textbackslash{}thetable\}\{\textbackslash{}arabic\{table\}\}}
\textcolor{magenta}{Defines table labels as Arabic numbers}\\
\texttt{\textbackslash{}setcounter\{figure\}\{0\}}
\textcolor{magenta}{Set figure number to 0; first figure will be Figure 1}\\
\texttt{\textbackslash{}setcounter\{table\}\{0\}}
\textcolor{magenta}{Set figure number to 0; first table will be Table 1}

\vspace{.5cm}

We do not need to reset the page numbers because they continue from the
Executive Summary. The other place you need to reset page and figure
numbers is with each appendix:

\texttt{\#Appendix\ A.\ Appendix\ Title\{-\}}
\textcolor{magenta}{Appendix header without a number}\\
\texttt{\textbackslash{}label\{sec:AppendixA\}}
\textcolor{magenta}{Creates a section label to reference it throughout the document}\\
\texttt{\textbackslash{}renewcommand\{\textbackslash{}thepage\}\{A-\textbackslash{}arabic\{page\}\}}
\textcolor{magenta}{Precede page numbers with A-, e.g., A-1}\\
\texttt{\textbackslash{}renewcommand\{\textbackslash{}thefigure\}\{A\textbackslash{}arabic\{figure\}\}}
\textcolor{magenta}{Add A to figures, e.g., Figure A1}\\
\texttt{\textbackslash{}renewcommand\{\textbackslash{}thetable\}\{A\textbackslash{}arabic\{table\}\}}
\textcolor{magenta}{Add A to tables, e.g., Table A1}\\
\texttt{\textbackslash{}setcounter\{page\}\{1\}}
\textcolor{magenta}{Set the first page number to 1}\\
\texttt{\textbackslash{}setcounter\{figure\}\{0\}}
\textcolor{magenta}{Set the figure number to 0; first figure will be Figure A1}\\
\texttt{\textbackslash{}setcounter\{table\}\{0\}}
\textcolor{magenta}{Set the figure number to 0; first table will be Table A1}

\vspace{.5cm}

The appendices for the SS code do not include commands to reset figure
and table numbering, as there are no figures or tables in these
sections.

\subsection{Lists}\label{lists}

I prefer using LaTeX for lists, but
\href{http://rmarkdown.rstudio.com/authoring_basics.html}{R Markdown}
also has its own syntax for lists. A LaTeX example is below:

\begin{Verbatim}[frame=single]
Ordered lists:
\begin{enumerate}
  \item List item No. 1 in the list
  \item List item No. 2 in the list, etc.
\end{enumerate} 

Unordered lists:
\begin{itemize}
  \item First item
  \item Next item
\end{itemize} 
\end{Verbatim}

\subsection{Equations and math mode}\label{equations-and-math-mode}

\href{https://en.wikibooks.org/wiki/LaTeX/Mathematics}{Wikipedia} and
the internet can help you with mathematical symbols as needed.

To get subscripts, superscripts and degree symbols for
latitude/longitude, the easiest way is through math mode. In LaTeX,
dollar signs indicate inline math mode, the underscore produces a
subscript, and a caret produces a superscript. You'll see throughout the
document that biological reference points are typed in math mode,
\verb|$SPR_{50\%}$| is \(SPR_{50\%}\). As noted earlier in the document,
you must precede a percent sign with a backslash when typing in math
mode.

For latitude or longitude, follow the format
\verb|$40^\circ 10^\prime$|, which produces \(40^\circ 10^\prime\). As
discussed above in the Tables section, when you use math mode within R
code to create a table, two backslashes are necessary, and the latitude
or longitude is written as \verb|$40^\\circ 10^\\prime$|.

\subsection{R code chunks}\label{r-code-chunks}

R code is written in the .Rmd file as R code chunks. R code can be
rendered or displayed for illustration and the
\href{https://www.rstudio.com/wp-content/uploads/2015/03/rmarkdown-reference.pdf}{R
Markdown Reference Card} contains almost everything you need. Here is
the simplest R code chunk, which also prints the results:

\begin{Verbatim}[frame=single]
      
      ```r
          1+1
      ```
      
      ```
      ## [1] 2
      ```
\end{Verbatim}

The preamble of Assessment\_template.Rmd sets the global options for R
code chunks

\begin{Verbatim}[frame=single]
<!-- #commented out these lines for presentation puposes
` ```{r global_options, include=FALSE} #sets global options 
  knitr::opts_chunk$set(echo=FALSE, warning=FALSE, message=FALSE) #options
```  -->
\end{Verbatim}

These options tell R Markdown not to print the R code (echo=FALSE), and
to ignore warnings and messages from R (message=FALSE and
warning=FALSE). The additional code chunk option of results=`asis'
appears in R code chunks for tables to prevent unwanted reformatting.
See the
\href{https://www.rstudio.com/wp-content/uploads/2015/03/rmarkdown-reference.pdf}{R
Markdown Reference Card} for more details.

\newpage

\subsection{Commenting}\label{commenting}

Comments within the main R Markdown document body are included via HTML
syntax, \(<!-- \text{Add your comment here} --!>\)

\textbf{Inline R code} within an HTML comment is still rendered.
However, you can comment out R code chunks.

\begin{Verbatim}[frame=single]
If you type:
<!--R code within an HTML comment that is NOT commented will
be evaluated, 'r 1+1'. -->

This is the result:
<!-- R code within an HTML comment that is NOT commented will 
be evaluated, 2. -->

If you type:
<!-- R code within an HTML comment that IS commented out  
produces a blank space, 'r #1+1'. -->

This is the result:
<!-- R code within an HTML comment that IS commented out  
produces a blank space, . -->

\end{Verbatim}

\subsection{ADA compliance}\label{ada-compliance}

In 1998 Congress passed the Section 508 Amendment to the Rehabilitation
Act of 1973, requiring that all federally-funded documents are
accessible to those with disabilities. Andy Clifton has the
\href{https://github.com/AndyClifton/AccessibleMetaClass}{AccessibleMetaClass}
on his GitHub page. The accessibility-meta.sty is part of the
Assessment\_template.

Alternative text is a description of an equation, link, or figure. These
are pop-ups in a PDF viewer, i.e., hover your mouse over the picture and
the pop-up will appear. These can be added to the source document using
\textbf{pdftooptip} from the \texttt{pdfcomment} package. For example,
the cover photo is now included using
\texttt{\textbackslash{}pdftooltip\{\textbackslash{}includegraphics\{cover\_photo\}\}\{This\ is\ a\ fish.\}}.

To create the ADA compliant PDF, you'll need to take the final TeX
document and compile it outside of R Markdown. In the TeX preamble add
the following
package.\texttt{\textbackslash{}RequirePackage{[}l2tabu,\ orthodox{]}\{nag\}}.
currently, there is an error that will require a lot of time to debug
(!TeX capacity exceed, sorry\ldots{}). However, the document can be
tagged in Adobe Pro in a matter of seconds.

\subsection{Common Errors}\label{common-errors}

\begin{itemize}
\item
  You have a version of the PDF assessment document open (not just the
  preview version)
\item
  You forgot a backslash before an underscore, or another reserved LaTeX
  character
\item
  Your table alignment is off, or xtable just doesn't like your
  alignment parameters. Simplify the alignment and add in alignment
  justifications incrementally
\item
  You have too many tables or figures in a row without a float barrier.
  You need to insert `\textbackslash{}Floatbarrier' between table or
  figure, Error:``LaTeX Error: too many unprocessed floats.''
\item
  You have a space in a file name somewhere. LaTeX does not like to read
  file names with spaces.
\item
  You have an illegal space or carriage return, most likely in the R
  Markdown image syntax
\item
  You don't have a blank line after an R code chunk. \emph{Sometimes},
  you'll get an error if you end an R code chunk and on the next line
  include a float barrier or other LaTeX syntax.
\end{itemize}


\end{document}
